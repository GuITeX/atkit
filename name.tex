% File di esempio per la classe 'arstexnica'
% Sample file for 'arstexnica.cls'.
\documentclass{arstexnica}

% Il codice che segue rende possibile usare uno qualsiasi tra i
% principali motori di tipocomposizione pdfLaTeX, XeLaTeX, LuaLaTeX.
% Si prega di non modificare il preambolo se non è strettamente
% necessario.  È consentito aggiungere altre lingue e font particolari
% ove l'argomento dell'articolo lo richieda. Altre personalizzazioni
% vanno fatte nei file `\jobname-package.tex' e
% `\jobname-command.tex'.
%
% The following code allows the use of any typesetting
% engine among pdfLaTeX, XeLaTeX, or LuaLaTeX.
% Please, don't change the preamble unless it is strictly necessary.
% You can add other languages or fonts if it is required by
% the subject of the paper. Other customisations should be added to
% the files `\jobname-package.tex' and `\jobname-command.tex'.
%
\ifbool{PDFTeX}{%                                    pdfLaTeX
    \usepackage[T1]{fontenc}
    \usepackage[utf8]{inputenc}
    \usepackage[english, italian]{babel}
  }{\ifbool{XeTeX}{%                                 XeLaTeX
    \usepackage{polyglossia}
    \setmainlanguage[babelshorthands]{italian}
    \PolyglossiaSetup{italian}{indentfirst=false}
    \setotherlanguage{english}
    }{\ifbool{LuaTeX}{%                              LuaLaTeX
      \usepackage{polyglossia}
      \setmainlanguage[babelshorthands]{italian}
      \PolyglossiaSetup{italian}{indentfirst=false}
      \setotherlanguage{english}
      }{%
      \ArsTeXnicaError\endinput
      }
    }
}

\usepackage{cochineal}
\ifbool{PDFTeX}{%
    \usepackage[varqu,varl,var0]{inconsolata}
  }{%
    \fontspec[StylisticSet={1,2,3},Scale=MatchLowercase]{inconsolata}
}
\usepackage[scale=.9,type1]{cabin}
\usepackage[cochineal,vvarbb]{newtxmath}
\usepackage[cal=boondoxo]{mathalfa}

\usepackage{microtype}
\usepackage{natbib}
\usepackage{graphicx}
\usepackage{hyperref}

% Inserire qui tutti i pacchetti aggiuntivi a quelli usati nel master, 
% necessari al proprio lavoro.
% Possibilmente includete un solo pacchetto per comando \usepackage,
% come nell'esempio che segue.

% Put here all the packages you need, other than those included in the master
% file.
% Please, include only one package per \usepackage, as in the following
% example.

% \usepackage{curve2e}
% \usepackage{fancyvrb}

% Inserire qui tutti i comandi definiti dall'autore per il proprio articolo.
% Meno ne definite, meglio è. Cercate prima un pacchetto che ne definisca uno
% uguale.
% Put here the author's commands, defined to typeset the article.
% The less commands, the better. Before defining a command, look for a
% package that provides you with the same command.

\newcommand{\qq}{QQ} % Replace it with your command(s). Delete it if you don't
		     % have any command.

\begin{document}

% In questo file scrivete solo il vostro articolo. Non mettete comandi o 
% includete pacchetti: ci sono file dedicati a ciò.
% L'articolo deve essere interamente racchiuso all'interno
% dell'ambiente 'article'.

% Put in this file your paper only. Don't put in it commands nor include
% packages: there are files for those purposes.
% The paper must be enclosed within the 'article' environment.

\begin{article}

\title[Titolo breve-Short title]{Titolo Lungo-Long title}

% Ripetere le informazioni per ciascun autore.
% Repeat info for each author.
\author{Autore (Nome Cognome)/Author (Name Surname)}
\address{Indirizzo o affiliazione-Address or affiliation}
\netaddress{Author's email address}

\maketitle

\begin{abstract} % Write here the abstract in the main paper's language

\end{abstract}

\begin{otherlanguage}{} % Second language
\begin{abstract}

\end{abstract}
\end{otherlanguage}

% La classe mette a disposizione dell'autore i seguenti comandi:
% The class provides us with the following commands:
% \cmdname{command} -> \command
% \meta{arg} -> <arg>
% \clsname{class}
% \pkgname{package}
% \envname{environment}
% \optname{option}

% Figure e tabelle devono, di preferenza, essere incluse in un
% ambiente flottante e munite di didascalia. Eventuali figure esterne
% devono essere inviate insieme al file .tex, in un formato
% compatibile con pdftex (PDF, JPEG, PNG).
% Figures and tables should be enclosed in a floating environment, and
% come along with a caption.
% External figures must be sent along with the .tex file, in a
% format suitable to be compiled with pdftex (PDF, JPEG, PNG).


% Bibliografia: l'autore deve accludere il file .bib (preferibilmente) o il
% .bbl contenente la % bibliografia relativa all'articolo.
% Bibliography: authors must include paper's bibliography (.bib, possibly, or 
% .bbl) files.

\bibliographystyle{arstexnica}
\bibliography{<file bibliografia>}

\end{article}

\end{document}
